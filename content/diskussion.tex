
\section{Discussion}

In Figure \ref{plot:CV} it can be seen that the shape approximatly matches with 
what is predicted by the Debye model. However, it is recognizable that there are
obvious deviations probably caused by measurement errors and instrument inaccuracies.
Further measurements would would make the curve more precise.
Aside from that, the theoretical 
\begin{equation*}
    \theta\idx{D,th} = 332,18\,\unit{\kelvin}
\end{equation*}
and the experimental Debye temperature
\begin{equation*}
    \bar{\theta}\idx{D,exp} = (347,11 \pm 55,6)\,\unit{\kelvin}.
\end{equation*}
was determined. 
According to the equation 
\begin{equation*}
    \frac{|\bar{\theta}\idx{D,exp}-\theta\idx{D,th}|}{\theta\idx{D,th}} \cdot 100\,\unit{\percent},
\end{equation*}
the two results overlap and they have a deviation of $4,5\,\unit{\percent}$.
Another point to be mentioned is that the measurement setup is not completely isolated from external influences since
the Dewar flask was open at the top resulting in thermal leakage.
In addition, manual adjustments of the voltage had to be made so that the sample and the shield have similar temperatures
which did not always work. For this reason, an appropriate suggestion for improvement is to adjust the voltage automatically.
Therefore, overall it can be said that the Debye model is roughly confirmed with this experiment.