\section{Goal}
In this experiment the specific heat capacity of copper as well as the debye temperature. Further the temperature dependence of the 
heat capacity will be compared to the theoretical predictions of the debye model.
\section{Theory}
There are 3 commonly used theories to describe the specific heat capacity. The classical approach, the Einstein model and the Debye model.
Each approach will be explained in the following.
\subsection{Classical predictions of the specific heat capacity}
In classical thermodynamics, the expected kinetic energy of a particle can be expressed as 
\begin{equation*}
    <u\idx{i}> = \frac{1}{2}k\idx{B}T
\end{equation*}
per positional degree of freedom of the particle, with $k\idx{B}$ being the Boltzmann constant and $T$ being the temperature. In generall
a particle has 3 positional degrees of freedom so a mean energy of 
\begin{equation*}
    <u\idx{i}> = \frac{3}{2}k\idx{B}T.
\end{equation*}
According to the Virial theorem the mean potential energy and mean kinetic energy of a harmonic oscillator is the same. The specific heat 
capacity of a solid, consisting of $N$ particles, is defined as
\begin{equation*}
    C\idx{V} = \frac{\delta U}{\delta T} = \frac{\delta 3\cdot Nk\idx{B}T}{\delta T} = 3\cdot Nk\idx{B}.
\end{equation*}
$C\idx{V}$ in this case is the heat capacity with constant volume. For solids it is often easier to measure the heat capacity at 
constant pressure, or $C\idx{p}$, which is related to the former via
\begin{equation}
    C\idx{p}-C\idx{V} = VT\frac{\alpha^2}{\beta},
    \label{eq:CpCV}
\end{equation}
with $V$ being the volume $\alpha$ the thermal expansion coefficient and $\beta$ the compressibility.
\subsection{Einstein model}
In the Einstein model, the quantization of vibrational energies and frequencies of individual atoms in a solid is considered. It assumes 
that all oscillators vibrate at the same frequency $\omega$, and their energies being multiples of $\hbar\omega$, where 
$\hbar$ represents the reduced Planck's constant. The probability, $W(n)$, of an oscillator having an energy equal to $n\hbar\omega$ 
at a given temperature $T$ is described by the Boltzmann distribution:
\begin{equation*}
W(n) = \exp\left(-\frac{n\hbar\omega}{k\idx{B}T}\right)
\end{equation*}
The average energy per atom, $<u>\idx{Einstein}$, is calculated by summing over all possible energies, $n\hbar\omega$, weighted by their 
respective probabilities, and dividing by the sum of all $W(n)$:
\begin{equation*}
<u>\idx{Einstein} = \frac{\sum_{n=0}^{\infty}n\hbar\omega W(n)}{\sum_{n=0}^{\infty}W(n)} = \frac{\hbar\omega}{\exp\left(\frac{\hbar\omega}{k\idx{B}T}\right)-1} 
\end{equation*}
The molar heat capacity at constant volume, $C\idx{V}$, can be calculated by differentiating the average energy with respect to temperature, 
like in the classical case, resulting in 
\begin{equation*}
C\idx{V, Einstein} = \frac{3R(\frac{\hbar\omega}{k\idx{B}T})^{2}\exp(\frac{\hbar\omega}{k\idx{B}T})}{(\exp(\frac{\hbar\omega}{k\idx{B}T})-1s)^{2}}.
\end{equation*}
For high temperatures it converges on the classical prediction of $3Nk\idx{B}$
\subsection{Debye model}
The Debye model takes into account the distribution of vibrational frequencies in a solid, rather than assuming a single frequency for all 
oscillators. It introduces a characteristic frequency called the Debye frequency, denoted as $\omega\idx{D}$, which represents the maximum 
frequency of vibrations in the solid. The energy levels of the vibrational modes are quantized in units of $\hbar\omega$, and the Debye model 
considers an integration over all possible frequencies weighted by the density of states function, $D(\omega)$. The average energy per atom 
in the Debye model, $<u>\idx{Debye}$, is given by:
\begin{equation*}
 <u>\idx{Debye} = 9Nk\idx{B}T \left(\frac{T}{\Theta\idx{D}}\right)^3\int_{0}^{\Theta\idx{D}/T} \frac{x^3}{\exp(x)-1} dx 
\end{equation*}
where $\Theta\idx{D} = \frac{\hbar\omega\idx{D}}{k\idx{B}}$ is the Debye temperature.

The molar heat capacity at constant volume, $C\idx{V}$, in the Debye model can be obtained by differentiating the average energy with 
respect to temperature:
\begin{equation*}
    C\idx{V, Debye} = 9R\left(\frac{T}{\Theta\idx{D}}\right)^3\int_{0}^{\Theta\idx{D}/T} \frac{x^4\exp(x)}{(\exp(x)-1)^2} dx
\end{equation*}
The Debye model provides a better description of the specific heat capacity of solids at low temperatures compared to the Einstein model. 
However, it also has limitations, such as assuming a linear sound velocity and neglecting anharmonicity effects.
 

